\documentclass{article}
\usepackage[utf8]{inputenc}
\usepackage{ctex}

\title{《逆熵 Anti-Entopy》番外|忘却的纪念\footnote{Click to View:\url{https://web.archive.org/web/20220331085224/https://mp.weixin.qq.com/s/ean3rDAEf3A4iCoBII4m4g}}}
\author{月下}
\date{2018-02-19}

% \setCJKmainfont[BoldFont = Noto Sans CJK SC]{Noto Serif CJK SC}
% \setCJKsansfont{Noto Sans CJK SC}
% \setCJKfamilyfont{zhsong}{Noto Serif CJK SC}
% \setCJKfamilyfont{zhhei}{Noto Sans CJK SC}
% \setlength\parindent{0pt}

\begin{document}
\CJKfamily{zhkai}

\maketitle


\Large


1 

瓦尔特叹了一口气,把欠条丢在了桌上,挫败感使他直不起腰来,思考让脑袋也变得沉重,他只能
用交叉的双臂勉强支撑着身体的重量。 

WELT Livehouse——就是他现
在经营的这个演出场地,已经数月无人光顾了。 

根本不用去分析和深究什么原因,因为这里根本连演出的乐队都没有,自然也不会有来看演出的观
众。 

瓦尔特现在回想起来,他从起步开始就是错的

\newpage
——不,或许连起步都是错的。 

堪称巨款的启动资金,是他好运爆棚中的天命
博彩公司的彩票。 

但接下来瓦尔特突发奇想要经营一家Livehouse——这种小规模的演出场馆形式据说来自
东方,他的运气就没那么好了。 


因为,他根本没有一点经营才能。 

瓦尔特笃信只要花重金来布置音响设备,乐队就会源源不断地被吸引而来,结果彻底忽略了宣传—
—根本没有人知道这里有个Livehouse。 

也有偶尔经过的路人出于好奇进来看一眼,但作为老板的瓦尔特毫无表演和唱歌的才能,在尴尬的
介绍和聊天之后,客人也只能选择离开。 

这么想着,瓦尔特叹了一口气,拿出一瓶珍藏许久的罗汉果利口酒倒至酒杯的三分之一处,又加入

\newpage
仅剩的一点威士忌,挤入几滴青柠汁。 


他抿了一口,然后噗地吐了出来。 


——原来自己调的酒那么难喝! 

怪不得上次那个客人喝了一口就放下酒杯,脸
色铁青地回去了。 

这么难喝的酒,借酒消愁就算了,但剑走偏锋,借酒壮胆还勉强有用。这时瓦尔特脑子里突然冒出
一个想法。 


2 


某大学内。 

“WELT Livehouse,如果你想
组乐队的话——” 

对面的路人学生扫都没扫他一眼,径直往前走

\newpage
去。 

放弃是明智之举,瓦尔特抽走传单,递给他面
前的下一位。 

“一流的音响设备,能同时容纳一千人的场地
——” 

这次的路人学生二号微微侧目,瓦尔特喜不自
禁地追了上去。 


“——一定能给你增加不少忠实的听众。” 

路人学生二号一边真诚地点点头,一边搂着同
伴——路人学生三号的手臂加快了脚步。 

瓦尔特突然冒出的想法,就是在大学里发Livehouse的传单,虽然学生并不是专业音乐人
士,但是却更加容易被说服。 


——自己早干嘛去了? 

\newpage


瓦尔特一边想着,一边暗暗锤了一下大腿。 

“我拿一张看看?”这时候,有一只纤细的手
伸了过来。 

瓦尔特忙不迭地把传单抽出来,递到说话的人
手里。 

“我们的场地费也不贵,就算是像你这样的学
生也能负担。” 


“学生?”拿着传单的手停了一下。 

察觉到对面的人好像发出了一声轻笑,瓦尔特
抬起了头。 

这是一个穿着米白裙子,留着一头浅绿色长发
的女子。 

“哎呀,这句夸奖我就收下了。”女子看向了

\newpage
手中的传单。 


她的声音很年轻,语调也很轻快. 


3 


办公室内。 

瓦尔特知道,就目前他所在的地点而言,自己
之前肯定是弄错她的身份了。 

普通教师?说不定是这座大学的教授也不一定
。 


一想到这里,瓦尔特就有些惴惴不安。 

“哦?你经营着一家Livehouse啊……”然而面前的女子却没有摆出任何架子,仔细地研究传单后露出了一个略带惊讶的表情,又抬起头看了
瓦尔特一眼,“你这么年轻,难道还是学生吗?” 

“这么年轻出来创业真不容易,现在的孩子啊
\newpage
……”女子感叹道,“虽然各自都有很多不同的想法
,但面对的艰难都是相同的。” 

——对方同样猜错自己的年龄了,但瓦尔特并没有否认。这样错误的判断对他十分有利,虽然对于
利用对方的善意他有些不好意思。 

这时,办公室的门发出“吱呀”一声,瓦尔特
转过头,看到一个穿着西装的女孩子正站在门口。 

看到瓦尔特,她皱了皱眉,目光瞟向了瓦尔特
身后。 


“普朗克,你又捡了什么东西回来?” 

被称为“东西”的瓦尔特赶紧低下了头,没有
说话。 

被唤作普朗克的女子则完全无视了这个问题,
走过去把传单递给了她。 

\newpage

“爱迪生,看看有什么能帮他的,这对你来说
是小意思吧。” 

瓦尔特偷偷抬了抬眼,这个名叫爱迪生的女孩子看上去好像略比普朗克年轻一些,一头海蓝色的长发清爽地束在脑后,服帖的西装恰到好处凸显出了她挺拔的身姿,这让穿着不合身的租赁西装的瓦尔特再
一次低下了头。 

“Livehouse……”爱迪生扫了两眼
传单,“弄些顶尖的音响设备就可以了。” 

瓦尔特连连摆手,“音响设备我都已经有了。
” 


“什么型号?”爱迪生挑了挑眉。 


瓦尔特老实地一一报上来。 

爱迪生听完,面无表情地说:“普朗克,电话

\newpage
借我用下。” 

然后瓦尔特听她熟练地报出了一长串自己曾经
视为“天价”的设备和乐器型号。 

“已经照着这个地址送去了,有些东西这里暂时没有库存,不过最晚后天也能到了。”爱迪生放下
了电话,对瓦尔特说道。 

瓦尔特使劲抹着汗,“那我……我也不能白拿
这些东西!” 

“虽然我不知道你这来路不明的家伙对普朗克说了什么花言巧语。”爱迪生露出了一丝轻蔑的笑容,“但是如果你不拿出点真本事来的话,这些东西的
钱也够你还一辈子了。” 

事到如今,瓦尔特也只得硬着头皮毕恭毕敬地
问道:“请问我要做什么?” 

“去说服特斯拉来你的Livehouse表

\newpage
演,这笔账就一笔购销。” 


瓦尔特惊讶地抬起头:“特斯拉?谁?” 

“对,特斯拉。”爱迪生脸上不知为何蒙上了杀气,“一个不自知的笨蛋,脾气很臭还有暴力倾向。一把年纪了还要扎高双马尾,让她扎单马尾还要和我吵架,不然就威胁我炸实验室。以及审美堪忧,明
明是红发还要戴红框眼镜和红色斗篷。” 

爱迪生吸了一口气,“总而言之,是一个不在
视线范围内就让人非常不安的家伙。” 

瓦尔特抓了抓头发,“那……为什么要找‘这
样’的人来?” 

爱迪生挑起一边的眉毛:“为什么?你不需要知道为什么。这对你来说也是好事吧,我可是好心好
意告诉你该找谁来演出。” 

“别随便牵扯进我的学生啊。”普朗克叩了叩

\newpage
桌子,“你知道她是故意躲着你的吧。” 

“就是因为这样才好久没见她了。”爱迪生收起‘杀意’,歪着头,露出一脸无辜的笑容,“舞台
上的再会真令人期待啊。” 

“如果是这样的话,那我也得跟着去了。”普朗克抱胸坐在了椅子上,“保障我的学生人身安全可
是我的职责。” 

“普朗克,Livehouse是给年轻人挥洒汗水,玩时下最流行的名为摇滚的音乐,不是你那
高雅的‘小提琴’能登场的地方。” 

瓦尔特想起了普朗克纤细的手指,放在小提琴
上确实很般配。 

而面对爱迪生一脸略带揶揄的笑容,普朗克也
报以胸有成竹的微微一笑。 

“对了,关于你刚刚提到的那个人…呃是叫特斯拉来着…?”瓦尔特趁着两人对话的间隙,小心翼
\newpage

翼地问道。 

“哦,说到我这个可爱的学生。其实我也知道她偷偷练贝斯很久了,但是却没有人邀请她上场表演。”普朗克朝瓦尔特眨了眨眼睛。“而且,你千万不能说破这一点,她一定会觉得你是在可怜她,不仅会
拒绝你,还有可能暴打你一顿。” 

普朗克想了想,又补充了一句:“而且千万不要说是爱迪生找她的,不然她一定会拒绝你,也一定
会暴打你一顿。” 

瓦尔特不禁感到脊背一阵发凉,但是,这时他
已经不能回头了。 

普朗克拿起笔,在纸上飞速地写下了什么,“
这是特斯拉的临时住址。” 

她折起纸,递给瓦尔特时,又若有所思地补充了一句:“对了,除了特斯拉,那里还有一个人。那个人在的话,你的Livehouse一定会爆满。
\newpage
但是你不要指望能说服那个人,那个人的来去全凭她
自己的想法。” 


瓦尔特愣了一下。 

“别紧张。”普朗克一手搭上了瓦尔特的肩膀,凑到他的耳边轻声说,“我觉得你们之间的缘分或
许还不错?” 


4 

现在,瓦尔特战战兢兢地站在这栋小洋房面前,从这里的位置来看,离闹市只有步行距离,却又闹
中取静,深藏在层层绿意中。 

他按响了门铃,报上了普朗克的名字并说明了来意,出人意料的是对方并未过多的盘问便开了门。
 

瓦尔特刚刚松了一口气,但是一进门,他就感

\newpage
受到了眼镜片后挑剔的目光。 


还真的是红框眼镜。 

如爱迪生所说的那样,扎着双马尾的特斯拉歪
着头,正一脸嫌弃地盯着她。 

“去你那种场地表演?明明是快要倒闭了吧,
我可没兴趣。” 


瓦尔特连忙摆手。 

“不不……该怎么说呢,比起说这是普通的摇滚演出,不如说是专业级的对战比较好。”一看到特斯拉拒绝了他,瓦尔特有点慌了神,“现在的观众对摇滚演出非常挑剔,如果只是单纯炒热气氛,他们也
不会抱有太大的兴趣。” 

“那让我一个人贝斯solo也没什么意思吧
?!”特斯拉不满地发问。 

“不不,你看……观众看多了吉他solo,
\newpage
所以贝斯solo才更有新鲜感,也更能体现出弹奏
水平……” 

瓦尔特其实也不太明白自己说的话到底是什么
意思,但只要听上去足够唬人和吹捧就行了。 

“而且在看过你出色的单独演出后,之后还有
同样专业的人才同场竞技,演出才更具可观性。” 

“哼,明明只有我一个人就够了。”特斯拉往前跨了一步,戳了戳瓦尔特的胸,“你是看不起我吗
?” 

瓦尔特连连摆手:“不不,不是说你一个人不够撑起场面,而是你与其他人之间擦出的火花更能激
发观众观看的欲望。” 

“哦?你是说这样的火花吗?”这时,从里屋
传来了一个懒懒散散的声音。 

一个蓝色短卷发的少女穿着一边长一边短的白
\newpage
衬衫,赤脚走了出来,她径直走向特斯拉旁边,勾住
了她的肩膀。 

瓦尔特也不禁心动了一下,刚想点头表示这个示范动作做得不错,特斯拉却一把把蓝发少女推开。

“喂……你这满是起床气的人碰我干嘛?!”

被推开的蓝发少女没有生气,反而笑着说:“我可是一大早就起来工作了,倒是你,是听到门铃以
后才慌慌张张地穿好衣服下楼的吧? 

“瞎……瞎说什么?!”特斯拉赶紧去捂住蓝发少女的嘴,“那你听到门铃响为什么不去开门?”

“因为我在工作,工作不能被打断啊。”蓝发少女狡黠地眯了眯眼,“况且,我在楼上的窗户看了一眼,这个人紧张地搓着手来回踱步的样子也真有意
思。” 


\newpage

“呃……”刚回过神来的瓦尔特一时语塞。 


紧张之下,瓦尔特只能看向蓝发少女。 


“朝哪里看啊你?!” 


特斯拉不耐烦地朝他翻了一个白眼。 

已经快要大脑短路的瓦尔特突然想起了普朗克
临走前和他说的话。 


“那里还有一个人……” 


应该指的就是这位蓝短卷发的少女了吧。 

“有时间在那里瞎看,还不如快帮我找一把贝
斯!”特斯拉又拉了瓦尔特一把。 

“你…决定去参加演出了?”回过神来的瓦尔特大喜过望,又转念一想,“什么?!你练贝斯的却
没有贝斯?!” 

\newpage

“少废话!”特斯拉踢了他一脚,“我没有贝
斯的话,你应该比我更着急吧!” 

这个人真的没问题嘛……瓦尔特不禁开始担心起来,于是又把求救的目光转向了看似心不在焉的少
女。 

“那个……这位小姐……不如你也一起来吧?

“喂,你又在和爱因斯坦瞎说什么呢?!”本来已经准备离开的特斯拉听到瓦尔特的话,又折了回
来。 

那个名叫爱因斯坦的少女抓了抓头发,“这点
小事,你们自己就能搞定吧。” 

瓦尔特以为她终于意识到要将自己乱糟糟的头发理顺一点,没想到爱因斯坦抓了以后,头发显得更蓬乱了。抓完头发,她又朝瓦尔特微微一笑,仿佛她
早已了解这个动作是他这样的凡人所不能理解的。 

\newpage


5 

瓦尔特站在WELT Livehouse里的舞台上,望着台下黑压压的人群,感觉有点眩晕。


这种不真实感,方才在后台就开始蔓延了。 

——当特斯拉一边给贝斯调着音,一边咕哝着
这么穷酸的场地怎么会有这么顶尖的贝斯。 

瓦尔特当然只能苦笑不语,要是知道这把贝斯是爱迪生送来的,恐怕有五个自己都不够特斯拉揍的

而此时,当瓦尔特介绍完普朗克是第一个表演嘉宾下台后,他仍旧不忘从台下寻找爱因斯坦的身影

虽然并不知道这个少女有什么才能,但是在他
心中,她懒洋洋的笑容却仿佛是一根定心针。 

一席白衣的普朗克缓缓走上台站定,架起小提琴,清亮的声音便从弓间跃出。相比起其他乐器,小
\newpage
提琴的音色更加近似人声。特别是瓦尔特听说,小提琴的音色还会越来越和主人的性格靠近,即使是同一把琴,则会随着主人的喜好渐渐磨炼出不同的声音。因此,说每把小提琴的声音都是独一无二的也不为过

瓦尔特观察着场下的人群,有一部分应该是那座大学的学生,特地来此捧普朗克的场。但剩下的大多数观众,应该是看到瓦尔特贴在Livehouse外以及街道上的海报前来看演出,现在则很是疑惑

毕竟,Livehouse里上演的应该是摇滚演出,而不是这样的“Concert”(演奏会
)。 

而且,即使是瓦尔特自己,也开始觉得普朗克的演奏变得索然无味,只是来来回回在几个双音音阶
中徘徊。 

有些观众终于也按捺不住发出了嘘声,很多人站起来大喊“滚下去,这不是摇滚!”,并纷纷往外

\newpage
走去。 

普朗克却没有因此而惊慌失措,而就在这时,一阵有序的鼓声从帷幕后响起,刚刚还准备离场的观
众纷纷停下了脚步,回过头来好奇地等待着。 

随着鼓声慢慢加快,普朗克的演奏速度也越来越快,白色长裙的裙摆随着她的左右晃动在台上翻飞着,而小提琴音色也一改之前的清亮,变得激昂起来

但是,尽管普朗克不甘示弱地紧跟鼓声的节拍,但鼓声始终比小提琴要快个小半拍。瓦尔特与观众都紧张地盯着小提琴,生怕弦就会在下一秒就会断掉

然而就在那时,普朗克出人意料地把右手往旁边一甩,弓应声落在地上。她左手则把小提琴拿下肩膀,放在胸前,然后手指轻轻一转,小提琴在她手中转了好几圈,紧接着右手在琴身上一拍,然后就像弹
吉他般弹起了小提琴。 

但是只弹了几个音,普朗克的手指就被小提琴绷紧的弦割破,蹦出了一粒粒血珠,但她却仿佛浑然
\newpage

不觉,又和着鼓声坚持弹了半分多钟。 

沉闷的声音回荡在Livehouse中,观
众则几乎都傻了眼,然后又是一声巨响—— 


“哐!” 

普朗克弹完最后一个音,她的左手一松,小提
琴顺势砸向了地面。 

“现在就是摇滚了。”普朗克笃定地拍了拍手
,丝毫不顾血已经流到了手腕。 


寂静。 


这是当时唯一能形容台下发生的形容词了。 

没有欢呼也没有嘘声,不知道是受到了这场表演的震撼还是只是被吓到了,一些原来准备离场的观众陆陆续续地回到了座位上。而在瓦尔特看来,他们

\newpage
畏畏缩缩的样子更像是“受到了某种胁迫”。 

而导致这一切的“罪魁祸首”普朗克却满不在乎地将自己的右手随意包扎了几下,就拉着瓦尔特再次上场。在她的介绍下,瓦尔特才知道刚刚的鼓声来
自她的学生的演奏。 

随着帷幕缓缓拉开,鼓手的真容也逐渐揭晓。

这是一个戴着黑绿配色的猫耳帽,面无表情打
着鼓的单侧罗马卷少女。 


更加诡异的是,她还戴着一副眼镜。 

鼓声虽然密集,但是少女毫无变化的表情配上
过于可爱的帽子,总让人觉得提不起劲来。 


“猫耳和摇滚一点都不搭!”有观众喊道。 


“为什么要戴眼镜!” 


\newpage

“一点激情都没有!” 

“这也不是摇滚啊!”就像刚才一样,又有不
少观众唱起了反调。 

“跳出制约的框架,展现出个人的特色,这才是摇滚的本心。”瓦尔特只能硬着头皮向自己解释,
“很多人,恐怕都没有领会摇滚的真正含义。” 

“就让薛定谔,带你们了解真正的摇滚吧!”
普朗克自信地说道。 

“喵!”架子鼓旁边的立式话筒中发出了一声
清脆的喵叫,鼓声随即停止。 

瓦尔特不禁扶住了额头,“这……好像也不是
摇滚啊……” 

这时从舞台的后方,传来了一阵低沉的拨弦。刚刚还在窃窃私语发表不满的观众一下子又安静了下

特斯拉,戴着红框眼镜,穿着露肩黑边白色演
\newpage
出服,抱着贝斯登场,而那双显眼的红色连裤袜再次
体现了特斯拉独有的审美执着。 

终于来了。瓦尔特看到这样的场景,心里舒了一口气,以鼓声带领观众,再加入贝斯,是摇滚寻常的开场,看来这场演出终于要往正常的方向进行了。

“戴眼镜又怎么了?!”她向台下吼了一声,
“戴眼镜就不摇滚了吗?!” 


“是……”有观众稀稀拉拉地应答着。 

特斯拉闷哼了一声,走到薛定谔身边,伸手把
她的眼镜摘了下来。 

“那我偏要戴两副眼镜!”特斯拉把薛定谔的
眼镜架在了头顶上。 

瓦尔特不禁对这种无意义的抬杠行为再次扶住
了额头。 

\newpage

幸好,特斯拉没有再说什么,只是低着头抱着贝斯弹奏了起来,而低沉有力的和弦也一下子让观众忘记了眼镜的事情,沉浸在特斯拉的贝斯solo中

但是在一旁的瓦尔特仍然有一丝担心,那就是
…… 


台上传来了一阵破坏性的扫弦。 


台下的观众都纷纷捂住了耳朵。 


爱迪生背着吉他,一脸挑衅地看着特斯拉。 

“你……你怎么在这里?!”特斯拉先是一脸
震惊,然后冲着爱迪生大喊。 

“怎么样?那把贝斯用得还顺手吧!”爱迪生
的目光落在特斯拉手中的贝斯上。 

“难不成……这把贝斯……”特斯拉盯着贝斯,脸上青一阵白一阵的,开场前还如获至宝的贝斯,
\newpage

现在却像是一个烫手山芋。 

“来了,这就是我和你的决战时刻!”爱迪生
挑衅地向特斯拉勾了勾手指。 

而特斯拉这时却仿佛冷静了下来,“决战?就
凭你那连初学者都不够的水平?” 

“哦,原来如此,你就要用那把贝斯啊。”爱
迪生看似满不在乎地微笑着。 

这句话又戳到了特斯拉的痛点,她咬牙切齿地说道:“我今天就来告诉你,真正的强者根本不需要
用好的乐器来证明!” 

特斯拉一甩双马尾,右手扫过贝斯,但却没有发出任何声响。她看似忘情地弹奏着,但场上仍然只
有鼓声。 

瓦尔特一瞬间以为是话筒出了问题,当他准备冲上台上的时候才发现,特斯拉的手指根本没有碰到
\newpage

贝斯,她只是假装在弹而已,或者说…… 

“哦?居然放弃了自己的优势来弹空气贝斯?
”特斯拉意外的举动勾起了爱迪生的兴趣。 

“难道这就是手中无琴,胜似有琴?”普朗克
在瓦尔特边上若有所思地说道。 

“可是她手里,明明有贝斯啊!”瓦尔特想大
喊一声,却又什么都没说。 

“能做出这样表演的人,说明她心中真的有摇
滚。”普朗克感慨道。 

“不……所以说……”瓦尔特已经不知道该做
出怎样的评价了。 

这时,爱迪生一脸释然的样子说道:“那我就
陪陪你好了。” 

她也开始表演起了空气吉他,疯狂地开始扫弦
\newpage

,当然——根本没有发出任何声音。 

瓦尔特倒是默默庆幸,如果爱迪生以刚刚那个
样子开始弹奏,观众恐怕会立马走人。 


但其实现在的状况,也好不了多少。 

两个人在台上无声的演出仿佛群魔乱舞一般,台下的观众抱怨的、收拾东西的、准备举起手中的橘
子皮往台上丢的,应有尽有。 


“果然这个场地注定是要倒闭的吧……” 

像是失去了所有的希望,瓦尔特慢慢闭上了眼
睛。 

这时,有一阵轻轻的哼哼声从瓦尔特的身边飘
过。 

瓦尔特刚刚来得及抬眼,白色的身影已经轻盈地跳上了舞台,少女白色外套领子上的方形装饰折射
\newpage

了光线,倏地一闪。 

还是平时蓬乱的蓝卷发,但在这时倒显得与气氛十分相符,平时为了防止刘海遮住眼睛的发夹也拿
掉了,额发自然地垂向两边。 

但是这时,外貌已经是次要了,因为声音已经先一步抓住了人们的耳朵。从爱因斯坦口中倾泻出的
声音并不响亮,却一下子穿透了人心。 

原来有些机械的鼓声,像是在歌声的带动下,突然有了错落的节奏。然后反应过来的是特斯拉,她
左手按住贝斯指板,右手开始慢慢拨弦。 

有了好的贝斯,一首乐曲仿佛就有了骨骼,而
贝斯和鼓则可以搭上任何另一个乐器。 

“喂你,吉他弹不好的话就换键盘吧!”特斯
拉朝爱迪生喊道。 

“不要小看我啊!”爱迪生调整了一下姿势,
\newpage

扫出了几个简单的和弦,慢慢地加入进来。 

不知是谁带领着谁,爱因斯坦的声音,也开始
发生了变化。 

还是那个清澈的声音,却因为稍稍改变了发声方式,音色略微沉一些,增添了一种金属的质感,也
更加摇滚起来。 

爱因斯坦走到特斯拉身边,先是搭着她的肩,两个人相视而笑,然后爱因斯坦转过身靠着她的背后
,两个人的配合十分默契。 


下面的观众也欢呼起来。 

“太好了……”瓦尔特不禁舒了一口气,看来这场压轴的演出终于能给WELT Livehou
se赢回人气了。 

这时,台上的爱因斯坦微微朝他瞥了一眼,不

\newpage
知是不是瓦尔特的错觉,她好像朝他笑了。 

但是这一切,又迅速湮没在欢呼的浪潮声中。


6 


1982年10月。 

“哎……为什么又让我来整理啊,直接扔掉就
好啦。” 

特斯拉叹了一口气,蹲下来随手翻着地上那一
叠叠旧报纸。 

这个房间是用来堆旧物的,已经有一段时间没人进出了,特斯拉总觉得空气中有股奇怪的味道,于
是打开了窗来通风。 

“咦?这个是什么?”特斯拉眼尖瞥到了一堆印刷品中的异物,那是压在旧报纸下的几张写得密密
麻麻的纸。 

\newpage

“爱因斯坦的笔迹……这是推导公式的草稿么
?”特斯拉抽出了其中一张快速地扫了一眼。 

特斯拉看着看着,拿着纸的手居然颤抖了起来


“这……这写的是什么小说啊!” 


她慌张地趴在地上,扒拉着其他手稿。 

“我什么时候做过这种没头没脑的表演啊!”

这时一阵风吹过,几张手稿飘到了一旁,底下
又露出了一张花花绿绿的纸。 

特斯拉急急忙忙抽出来一看,那是一张海报。


“逆熵乐队?!” 

“爱……爱因斯坦那个家伙!”看着海报上印刷的爱因斯坦,特斯拉一脸狂躁,“她到底在想什么

\newpage
啊?!还特地为了小说去做这种东西?!” 

这时,特斯拉才注意到了海报下方的“WEL
T”。 

她松开海报,看向了剩余的手稿。这时在她眼里,那个看上去是主角、却没什么戏份还很倒霉的家
伙的名字,在手稿里的各处慢慢清晰了起来。 

最终,她叹了一口气,慢慢抚平了刚刚被她不小心弄皱的海报,把它和其他的手稿整整齐齐叠在一
起,重新压回了旧报纸的底下。 


“哎……爱因斯坦那家伙。” 

特斯拉看向窗外,又是秋天,又是一年,树叶
毫不留情地落下。 

“或许明年,真的搞一下也不错?”特斯拉自
言自语道。 


\newpage

7 


1983年1月。 

“突然和我说要在逆熵总部年会上表演的时候我真是吓了一跳。”普朗克握着红酒杯,微微一笑,一头浅绿色的长发优雅地盘起,微微泛着灰白。“我
的学生还是这么真有活力啊。” 

台上蹦蹦跳跳正在表演的两人,正是爱因斯坦
和特斯拉。 

“是特斯拉姐姐——啊不,特斯拉的提议。”普朗克身边褐色短发,看上去约莫35岁上下的斯文
眼镜男接话道。 

“你还是改不了口啊,你现在这个样子,在街
上叫她们俩姐姐的话一定会被当做变态的。” 

褐色短发斯文男腼腆地笑了笑。“那是当然,
特别是现在这里有真正的年轻人加入。” 

\newpage


他们看向那个靠在墙边的深紫色长发男人。 

“雷电……龙马。”普朗克微微眯起了眼睛,“为了逆熵的壮大,我们的目光不能再局限在北美这
片土地上,必须招募世界各地的人才。” 

普朗克说完这句话的同时,特斯拉和爱因斯坦
也正好表演完一首曲子。 

等到所有人都再次围绕在自己身边,普朗克缓缓举起红酒杯致意:“干杯!为全新的逆熵,也为全
新的一年!” 

THE END...?

\end{document}
